%====================================================================%
%                  16LOMCON.TEX     October 2013                     %
% This LaTeX file has adapted various sources for the use in the     %
%      preparation of the standard Proceedings Volume                %
%====================================================================%

\documentclass[a4paper]{article}

\usepackage{16lomcon}        % Proceedings volume layout metrics
\usepackage{cite}             % Smart range citations
\usepackage{epsfig}           % Encapsulated PostScript figure inclusion
\usepackage{lineno}
\usepackage{atlasphysics}
%\usepackage{epstopdf}

%\bibliographystyle{atlasBibStyleWithTitle}
\bibliographystyle{unsrt}    % for BibTeX - sorted numerical labels by order of
                             % first citation.
\bibliography{stdenis}
%%%%%%%%%%%%%%%%%%%%%%%%%%%%%%%%%%%%%%%%%%%%%%%%%%
%                                                %
%    BEGINNING OF TEXT                           %
%                                                %
%%%%%%%%%%%%%%%%%%%%%%%%%%%%%%%%%%%%%%%%%%%%%%%%%%



%\newcommand{\affiliation}[1]{\author{#1}}
\newcommand{\sms} {\ensuremath{\sqrt{s}_{SM}}}
\newcommand{\spl} {\ensuremath{\sqrt{s}_{SM}}}
\newcommand{\grqs}{\ensuremath{\sqrt{s}_{GRQ}}}
\def\sg{\ensuremath{\sigma_G}}
\def\Hzz{\ensuremath{H\to ZZ}}
\def\Hww{\ensuremath{H\to WW}}
\begin{document}
\linenumbers
%%%%%%%%%%%%%%%%%%%%%%%%%%%%%%%%%%%%%%%%%%%%%%%%%%%%%%%%%%%%%%%%%%
% The preamble of the paper
%%%%%%%%%%%%%%%%%%%%%%%%%%%%%%%%%%%%%%%%%%%%%%%%%%%%%%%%%%%%%%%%%%

\title{Implications of the Higgs Boson Discovery}

\author{Richard St. Denis \email{richard.stdenis@glasgow.ac.uk}}

\affiliation{School of Physics and Astronomy, Glasgow University, G128QQ, Glasgow, UK}

% You may repeat \author and \affiliation as many times as necessary!

\date{}
% Print it out!
\maketitle
\begin{abstract}
  In this paper the significance of the discovery of
a new boson with the ATLAS detector at the LHC proton-proton
at a mass of
$$m_H = 124.3 ^{+0.6} _{-0.5} \rm {(stat.)} ^{+0.5}_{-0.3} 
\rm{(syst.)\, GeV}$$ is explored.  A number of computations that should
be rather simple are suggested and imply that this mass may well be a
fine-tuned value that is required for galaxy formation.  It is argued
that it points to the scale
 where new physics must exist and implies a range of possible new
physics symmetries over a variety of scales.  This leads to fundamental
questions regarding quantities such as action, entropy as well as fundamental
concepts of causality and an uncertainty principle regarding causality is
suggested.  A checklist of computable and verifiable computations is provided.
\end{abstract}

\section{Introduction}
The discovery of the Higgs is significant for a number of reasons, not the least of which
is the completion of the observation of the last of the fundamental particles in the Standard Model (SM). 
In this paper the significance of this value is considered and leads to a number of suggestions for 
how to cope with fine tuning as well as how to choose the fine tuning needed based on cosmological bounds.
The problem is illustrated by considering an experiment where collisions occur at a centre-of-mass energy
just above, below or equal to the planck mass. Consideration is then given to how one would realize it and
what the broad characteristics of the phenomnology would be. The role of black holes is then investigated and
it is proposed that there are different classes of black holes but that all black holes contain collision
scales that are compressed to that which is of order the Planck Mass, thereby providing an abundant number
of accelerators in which to observe the proposed experiment.  A new class of black hole is considered to be 
the instatiation of the Fermi sea of anti matter which contains energy and hence is dark matter. 
The paper is organized as follows. The signficance of the Higgs mass is considered in Section~\ref{sec:finetune}
and the fine tuning problem is outlined. The proposed experiment is described in Section~\ref{sec:collider}
where its phenomenology is also described and a connection to black holes is suggested and the melding 
relativisitic quantum field theory with a theory of general relativity and quantum mechanics is considered.
In Section~\label{sec:accelerator} consideration is given to how one might observe the phenomnology of
Planck Mass scale interactions. In light of cosmological arguments fine tuning is addressed in 
Section~\ref{sec:ftembrace}.  Consideration of the physics within the black hole is given in 
Section~\ref{sec:ftl} and this leads to a discussion of causality and a possible uncertainty principle
that sets the scale in which causlity may be violated.

\section{Significance of the Higgs Mass}
\label{sec:finetune}
There is a deeper significance to the discovery of the Higgs Boson: the value of the Higgs Boson mass itself.
The importance of this lies in the fact that the
SM is valid to scales of order the Planck mass but as one approaches
such scales, the vacuum becomes metastable~\cite{ref:instabilityPlanck}; however, for the SM to 
be a valid description of phenomena at this scale it is necessary to fine
tune the Higgs mass to a level of one part in $10^{52}$~\cite{ref:finetuning}.  
This feature of the SM is referred to as the naturalness problem~\cite{ref:unnat} and has motivated
the formation of theories beyond the SM to try to eliminate this fine tuning.   One of
the most popular theories to reduces this problem is Supersymmetry~\cite{ref:susy}. The theory is 
 usually cast in terms of the Minimal 
Supersymmetric Model (MSSM)~\cite{ref:MSSM} in order to allow for 
computations to be performed with a limited set of parameters as inspired
by assumptions of a Grand Unification Theory that includes gravity and
breaks down to SUSY at the electroweak scale, of order 1 TeV.  SUSY
is however a broken symmetry and as such there remains a fine tuning of
the Higgs fields in that theory, but to a much smaller degree than in the SM, being
of order $10^{-2}$ to ${10^-3}$~\cite{ref:SUSY-unnat}. Furthermore, the boson
that was discovered can be the CP-even scalar in SUSY since it is 
SM-like in all properties except for 20\% differences in cross section. Currently these
differences  can be 
easily accomodated within the present experimental
statistics~\cite{ref:crossSecSusyroom}.  However it is true that the 
loop corrections in SUSY that correspond to the tree level limit that the
lightest Higgs must have a mass less than that of the $Z$ become quite
unstable when accomodating a Higgs mass of order 125 GeV.

\section{Proposed Experiment}
\label{sec:collider}
  In either the case of the SM or the MSSM it is interesting to consider
an experiment that explores the SM processes that one typically measures
at a collider such as cross sections and differential measurements of
jets, boson production, Higgs production, heavy
quark production, but to consider these at scales approaching the Planck Mass. 
Specifically it is interesting to consider what is expected from an $e^+ e^-$ collider with
$\sqrt{s}= 10^{16}$ GeV?   Furthermore one could sweep a range of energies from
$\sqrt{s}_{SM} \equiv 10^{15}$ to $\sqrt{s}_{GRQ}\equiv 10^{17}$ GeV. 
The energy scan includes values that should be described by the SM to values
where the Planck Mass is exceed and the theory must include a relativistic
quantum field theory that incorporates General Relativity (GR). This is indicated here
by the subscript GRQ. Candidate models currently include string theory~\cite{ref:string}.


 While considering this experiment, one has to examine how the vacuum predicted
by the SM is changing with the scale. For values below $\spl$ and around
$\sms$ the vacuum has a minimum and states computed as excitations above
this minimum in the normal Relativistic Quantum Field Theory (RQFT) 
will lead to the effects that are normally understood within the context of
computations in perturbation theory although accuracy may demand rather higher
order computations in order to match observation.  As one approaches the point
where the vacuum has no minimum, then there is no broken symmetry and all
the bosons will be massless, like photons. The unbroken SM would then be
a basis for these phenomonelogical computations.  Once there is no minimum
for the vacuum for values of $\sqrt{s}=\grqs$ the framework of RQFT would
no longer be valid and a complete theory based on the computation of
QM with GR would be required.  As one does this energy sweep, it would
also be true that the tunneling probability from the region where the 
vacuum has a minimum to one outside that minimum would increase and the 
state may well be one in which the universe tunnels and the states are
free with no minimum.

\section{Cosmic accelerator}
\label{sec:accelerator}
It is entirely possible to consider the computations suggested for the collider experiment
proposed in Section~/ref{sec:collider} and to confront the possibility of tunneling and how that
might manifest itself.  If the energy input to a collision were to be dissipated by tunneling, then 
that would appear as black hole formation.  While it is impractical to perform the experiment 
it is possible that collisions of these energies could be observed in particle astrophysics. There
have been many examples where one could find phenomena that seem to be odd and rare.  Gravitational
lensing was considered an interesting thought experiment, unlikely to be found because it required a
very specific arrangement of galaxies and observers; however, this has been observed and indeed has
been used in measurements of MACHOS. It is commonplace. Also, supernovae were thought to be hard to 
find until systematic scanning of the sky using large data collection methods and CCD cameras has
led to the ability to find supernovae on demand and indeed to measure the accelerating expansion of the
universe.  Therefore in order to observe the collisions that are of sufficient energy that
 lead to tunneling  one needs to look for this.  

Black holes appear to come in two mass ranges.  One is that set by the death of stars having masses between
eight and fifty solar masses leading to black holes in this mass range.  The other is much larger, of order
billions of solar masses and these are found at the centre of galaxies. There is a black hole observed
in the globular cluster
% Eva Noyolla gimos at  Gemini observatory ESA nasa hubble space telescope Max Plank for extraterrestrial physics
 Omega Centauri that has a mass of 40,000 solar masses and could be of particular interest because of its
unique intermediate mass value.  There are also various generations of stars in Omega Centauri. It is thought
that supermassive black holes that are at the centres come from the merging of smaller black holes and that
these intermediate sizes are important to show the process of merging in place. However given the new information
on the mass of the Higgs, and considering the possibility of tunneling when collisions at the scale of the 
Planck mass occur, it may be that the larger black holes are in fact due to these tunneling phenomena rather than
the fusion of black holes. If one were to have fusion of black holes the spectrum of masses should be continuous
rather than discrete.  If collisions have occurred that cause a tunneling, then it may be that the process 
results in some characteristically larger size for black hole formation and indeed that as the tunneling
takes place, it may be possible to observe the tunneling process building up. 

This would require that there are simply cases of acceleration in the universe that can reach centre-of-mass
energies large enough for tunneling to occur. Alteratively the supermassive black holes may have been formed 
when the universe was young enough to have a high probability for collisions at high energy to occur.  This 
would have had to have happened just at the end of the inflationary period and implies another form of fine
tuning. It would have been necessary that tunneling probability not be too large such that every collision
would lead to black holes so that the universe would not get started; however, it would have to be large enough
to allow for sufficient black holes to be formed such that they produce the correct numbers of galaxies.
In addition the total mass of the universe should be large enough that these black holes do not condense and 
prevent formation of the universe at all.

\section{Fine Tuning Embraced}
\label{sec:ftembrace}
This implies a set of parameters for the formation of the universe: the probability of tunneling per 
high energy interaction, $P_{tunnel}$, the spectrum of particle center of mass collisions as a function of
the size and density of the universe $S\left(t,\rho\right)$ and the size of the tunneling $D\left(\sqrt{s}\right)$.
The tuning of these parameters is related to the Higgs potential and hence to the fine tuning of the Higgs mass.
As was stated in Section~\ref{sec:finetune}, the amount of fine tuning that is required for the Higgs may well
indicate if the particle needs to be SM or MSSM. 

If one considers the experiment in Section~\ref{sec:collider} then the theory to describe the full scan of
masses is that which accomodates Quantum Mechanics and General Relativity, a theory of GRQ. The energy scan 
is the same as a scan in size, and just as in atomic physics there must be a correspondence priciple where the
size or energy regime in which both theories are valid give a consisent answer. In this case the incomplete
theory is not Newtonian mechanics but RQFT based on Special Relativity and the more complete theory is not
quantum mechances but GRQ.   One would imagine that the dynamics of the black hole formation would be described
by  RQFT and set the parameters for the rate at which the tunneling occurs and the dynamics that limit the
tunneling such a to have some turn-off mechanism.  In addition it is possible that a structure of a RQFT GUT
would be set by the GRQ, with symmetries such as SO(10) or SU(5) predicted by the theory. From the point of view
of RQFT, it would appear as an ad-hoc fact that some symmetry and indeed the breaking of the symmetry is
described by some set of parameters that have to be set by hand.  In the breaking of the symmetry and any
further symmetries a set of Higgs bosons would be generated and of course a set of new particles.  Therefore 
it remains well worth pursuit of the search for these high mass particles throughout the scales up to the
Planck mass to try to get a clue on the form of the manifestation of the theory from which they follow.

In addition to this it is possible to formulate various ensembles of broken symmetries and hence fine tunings
that allow a range of fine tunings to be generated.  Thus one is not limited to only a fine tuning of the Higgs
mass to the large degree required by the Standard Model nor to the small degree of the MSSM, but could choose
tunings in a spectrum of values.   It is therefore proposed that a computation of the fine tunings for various
scenarios of symmetries being broken be examined to see what possiblities exist and how that then can be 
compared to the probabilities for the formation of black holes for galaxies.

In consider the formation of the black holes, it is also interesting to understand if these galaxies and their
black holes can eventually serve as portals to the next Big Bang or if indeed each of the provides
a transport of matter and energy to the formation of another Big Bang in smaller universes. The expansion
of the universe implies that these galaxies will not have an opportunity to coalesce and be the source of
the next Big Bang.  This would have happened if the universe were contracting and would provide a means by
which matter and energy could be transformed in the Big Crunch.

\section{Faster than Light Neutrinos and Physics Near the Planck Mass}
\label{sec:ftl}
While the faster than light (FTL) neutrino observation turned out to be the result of an instrumental error, it was
useful to think about what would one do to accomodate such an observation.  A fundamental assumption of
RQFT is causality and if one considers FTL neutrinos one should ask if there is a consistent field theory
where causality is formally abandoned.  It is possible to formulate such a theory~\cite{ref:QFTFTL} but the 
result is considered uninteresting since the particles of that theory cannot interact with slower than light
particles.  This however is considered in the context of RQFT rather than GRQ. Therefore it could be possible
that interactions could arise from gravity. Since causality is a statement regarding time, it is therefore 
necessary to abandon the need to consider time in FTL theory. Given that time slows down as one approaches
a black hole and stops at a black hole, it is possible to consider that the dynamics within a black hole allow
for time to be undefined and indeed correspond to scales smaller than that suggested by the Planck Mass. It is
possible that the black holes from tunneling are then different from those created by stellar collapse or that
the scale of the black hole is not characterized by the event horizon but necessarily by the Planck Scale. This
implies then that the physics at the Planck scale is much more common than one would otherwise 
suspect and that indeed collisions at that scale can be created in the stellar collapse. This in turn implies
that the formation of super massive black holes comes from stellar collapse seeds that then provide the
opportunity for a tunnelling black hole.

One does not have to abandon causality entirely but it may be that causality can be formulated 
in an uncertainty principle that requires abandoment for small distances only and that a new
constant analagous to Planck's constant can characterize the abandoment of causality.

It has been suggested that as one looks at scales near the Planck Mass~\cite{ref:wheeler} that space-time becomes
foamy.  It is natural to consider that at those scales causality can be abandoned.  If one were to try
to formulate a path integral where the path is foamy, then it is expected there would be issues with causality.
Therefore it is sensible to revisit the principle of least action and its definition once again, just as
was first done by LaGrange, later by Feynman who accomodated the Heisenburg uncertainty principle.  The issue
with causality becomes an issue of getting lost as one tries to integrate over the path. 

\section{Causality, Vacuum, Dark Matter and Dark Energy}
\label{sec:causality}
Having established that action needs to be redefined has a number of implications.  First, since the definition
of action is the difference in kinetic and potential energies, it is reasonable that once one considers action
in a GRQ theory, there could be terms that correspond in GR to the energy contained in the quantum foam which
in turn could be the definition of the vacuum energy above which excitations are computed in RQFT. The definition
of the vacuum in special relativity has been a difficulty. It is possible that dark matter and dark energy
are manifestations of the energy contained in this foam and are not related in any way to the MSSM. The 
interaction of particles of dark matter would indeed be entirely through gravitation and not require 
a RQFT formulation at all. By providing an energy density the infinities usually sidestepped in RQFT would
become finite since the size and density of the foam is set by the Planck Scale - one need only fill up the
universe with this background of foam.  The nature of the foam would be that of a third class of black holes that
in fact are the source of the negative energy sea of RQFT.  

\section{Summary}
In this paper it has been proposed to consider the investigations of particle astrophysics in light of 
the understanding of the implications of the Higgs mass.  An investigation of the phenomenology 
associated with collisions just below and above the centre of mass energy at the Planck Mass scale is 
suggeted.  It is also suggested to compute the possible fine tunings required for various scenarios 
of symmetry breaking.  With these numbers in hand, then it is worth considering how these phenomena would
be observed in cosmic accelerators and that the likelihood of such observations may be quite large.
The physics associated with this scan may also require a new interpretation of causality and the
vacuum to be studies may be one that describes the dark energy and dark matter of the universe as well
as being the vacuum upon which excitations of particles from RQFT is currently computed. Indeed the vacuum
is considered to be a third class of black holes that can provide a source of the Fermi sea of anti matter
through Hawking radiation and that matter arises from the evaporation of these black holes.  

  
\section*{Acknowledgments}
The author would like to thank the organisers of the conference for 
their hospitality and the intellectually stimulating atomosphere.


%%%%%%%%%%%%%%%%%%%%%%%%%%%%%%%%%%%%%%%%%%%%%%%%%%%%%%%%%%%%%%%%%%
% References
%%%%%%%%%%%%%%%%%%%%%%%%%%%%%%%%%%%%%%%%%%%%%%%%%%%%%%%%%%%%%%%%%%
% \begin{thebibliography}{99}
% \bibitem{art} A.~Author, B.~Author, Phys.~Rev. \textbf{D 51}, 153 (2005).
% \end{thebibliography}

%\bibliography{stdenis_bibliography}{}
%\bibliographystyle{Papers}

\end{document}

